As in every machine learning task that was more focused on checking whether the method may be applied to a specific problem rather than its results, a lot of improvement could be obtained by increasing the number of training and validation data and making it more diverse. This could for example cope with the problem of distinguishing the vertical stripe of the letter H with other narrow, red regions. Moreover, as in the system there is a lot of hyperparameters that influence different steps of the segmentation task only a few configurations of those parameters were tested, simply because it would take a lot of time to check them all. Hence, another possible improvement would be to perform an extensive search of optimal values of the hyperparamter set used in the developed system. 

The thesis was mostly focused on choosing the right algorithms that will allow to perform a semantic image segmentation without a need of extensive resources. As feature engineering is on itself a demanding and time-consuming task, in the thesis it was highly limited. Therefore, a lot of improvement can be achieved if proper features would be incorporated to the system. Basing on colour information only, it is not possible to perform semantic segmentation on natural images, as natural objects are usually not colouristically homogeneous as well as colours of objects of the same class tend to vary. Hence, the first possible extension to the system would be to increase a feature space by incorporating more complex features than just pixel colours.

Another possible extension of the thesis would be to use the developed system only as the postpocessing stage, after classification with some other algorithm takes place. Then, the initially segmented image obtain with this classification would become an input of Conditional Random Fields just like an image after colour quantisation, which was used in the thesis. State-of-the art systems are using Convolutional Neural Networks to define which regions should belong to which class, and Conditional Random Fields only to make those results better at object boundaries and to reduce noise being a side effect of initial classification. Hence, in the future work it may be tested whether the developed system would improve the segmentation results obtained by Convolutional Neural Networks or any other classifier. 