The part of the system that is described in this chapter was aimed to present a basic example on how Conditional Random Fields can be used to perform semantic segmentation on images. Therefore, only the basic features were chosen to represent the images. As it has already been explained in \textit{section \ref{sec:energy}: \nameref{sec:energy}} there are two types of features, those used for computing the unary potential, and those for pairwise potential. When it comes to the latter type, for this part of the system a binary pairwise feature was introduced. It is in a form of a two-dimensional vector that can take only two values: $[1,0]$ if the labels of a given pair are the same, or $[0,1]$ if they are different. Then, a feature function for a pairwise potential $\varphi_2$ can be computed as in formula \ref{eq:fi_pairwise_linear}, where the difference between labels of a given pair of output factors $\left | (y_i - y_j) \right |$ can be either 0 if labels are equal, or 1 if they are different. 
\begin{equation}
    \label{eq:fi_pairwise_linear}
        \varphi_2(y_i,y_j) = \begin{bmatrix}
        1 - \left | (y_i - y_j) \right | \\
        \left | (y_i - y_j) \right |
    \end{bmatrix}
\end{equation}

When it comes to the unary potential, only colour features of given superpixels were taken into consideration. Every  input factor can be described by three continuous numbers, each representing one colour component from a chosen colour space. As it was already explained, the unary potential is dependent not only on the chosen features, but also on the label, which is under consideration. As this part of the system performs classification into three distinct classes, the unary potential can be presented as in \ref{eq:e1_nonlinear}. 
\begin{equation}
    \label{eq:e1_nonlinear}
    E_1(y_i,x_i,w)= 
    \begin{cases}
        \left \langle w_{1,0}, \varphi({x_i}) \right \rangle , &  \text{if } y_i = 0\\ 
        \left \langle w_{1,1}, \varphi({x_i}) \right \rangle , & \text{if } y_i = 1\\
        \left \langle w_{1,2}, \varphi({x_i}) \right \rangle , & \text{if } y_i = 2\\  
    \end{cases}
\end{equation}
However, such representation is not very convenient from an implementation point of view. Therefore, instead of having three distinct weight vectors, a single weight vector can be proposed that will contain nine elements, as there are three colour components and three available classes. However, a nine-dimensional weight vector requires the feature vector to contain nine elements as well. Though colour features are independent from the assigned label, a feature function can be introduces that returns a different feature vector depending on a label. Such vector will contain one element for each colour component and six other elements with value equal to 0. Then, depending on the label that is under consideration, those three colour components will occupy a different section of a resulting vector. For label 0, colour components will be on indices 0-2, for label 1 on positions 3-5, and for last label on indices 6-8. Hence, a formula for a weight vector and a feature vector that were used in this part of the system is presented in equation \ref{eq:vectors_nonlinear}. Colour components are denoted as R,G, and B, tough different colour space than RGB can also be used. A value of a single element in a feature vector is dependent on the difference between a label of a current vector position, denoted by $y_0$, $y_1$, or $y_2$, and a label of the superpixel that is under consideration. This difference can take a value of 0 if labels are the same, or 1 otherwise.
\begin{equation}
    \label{eq:vectors_nonlinear}
    \begin{matrix} 
        w_1 = \begin{bmatrix}
            w_{R,0}\\ 
            w_{G,0}\\
            w_{B,0}\\
            w_{R,1}\\ 
            w_{G,1}\\
            w_{B,1}\\
            w_{R,2}\\ 
            w_{G,2}\\
            w_{B,2}\\
        \end{bmatrix}
        & & &
        \varphi_1(x_i,y_i) = \begin{bmatrix}
            \phi_{R} \cdot (1-\left | y_0 - y_i \right |)\\ 
            \phi_{G} \cdot (1-\left | y_0 - y_i \right |)\\
            \phi_{B} \cdot (1-\left | y_0 - y_i \right |)\\
            \phi_{R} \cdot (1-\left | y_1 - y_i \right |)\\ 
            \phi_{G} \cdot (1-\left | y_1 - y_i \right |)\\
            \phi_{B} \cdot (1-\left | y_1 - y_i \right |)\\
            \phi_{R} \cdot (1-\left | y_2 - y_i \right |)\\ 
            \phi_{G} \cdot (1-\left | y_2 - y_i \right |)\\
            \phi_{B} \cdot (1-\left | y_2 - y_i \right |)\\
        \end{bmatrix}
    \end{matrix} 
\end{equation}

Having specified how the feature functions for unary and pairwise potential it is possible to perform parameter learning and then inference process. As the weight vector for unary potential contains nine elements, and two elements for pairwise potential, in overall, it is necessary to adjust eleven weights. 