As it has already been presented, the created system is composed of two parts and this chapter will be devoted to the first one. It will involve an explanation of how the already described theoretical background can be adapted to perform the task specified in this dissertation. Implementation of methods chosen to address this task was based on a book titled "Structured Learning and Prediction in Computer Vision" by Sebastian Nowozin and Christoph H. Lampert \cite{Nowozin}, which provided a lot of essential information for the problem. The part of the created system that is described in this chapter was devoted to performing semantic image segmentation on a simple example of segmenting objects in an image based only on colours of those objects, without incorporating any contextual data. This part of the system was aimed to prove that with the use of Conditional Random Fields it is possible to perform semantic segmentation of simple images presenting objects that differ only by colour. The theory behind all the algorithms that were used to accomplish this task has already been presented in \textit{Chapter \ref{chapter:structured_prediction}: \nameref{chapter:structured_prediction}}.

The first section of this chapter will provide a detailed explanation of how the preprocessing stage was performed and why it is beneficial to incorporate it in the created system. The next section will describe the key difference between this chapter and \textit{chapter \ref{chapter:nonlinear}: \nameref{chapter:nonlinear}}, which is the way in which features are represented. Then, a dataset used for training and testing of the solution will be presented. The next section will provide information on how the results of semantic segmentation will be evaluated. Last, but definitely not least, an overview of the conducted experiments and their results will be provided.
