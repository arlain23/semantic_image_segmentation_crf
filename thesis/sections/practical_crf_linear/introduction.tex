As it was described, the created system is composed of two parts. In this chapter the first one will be described. It will involve an explanation on how the theoretical background that was already explained can be adapted to perform the task specified in this dissertation. Implementation of methods chosen to address this task was based on a book titled "Structured Learning and Prediction in Computer Vision" by Sebastian Nowozin and Christoph H. Lampert \cite{Nowozin}. The part of the created system that is described in this chapter is devoted to perform semantic image segmentation on a simple example of segmenting objects in an image basing only on colours of those objects, without incorporating any contextual data. This part of the system was aimed to proof that that with use of Conditional Random Fields it is possible to perform semantic  segmentation of images, which can also contain some noised data. The theory behind all the algorithms that were used to accomplish this task has already been presented in \textit{Chapter \ref{chapter:structured_prediction}: \nameref{chapter:structured_prediction}}.

First section of this chapter will provide a detailed explanation on how the preprocessing stage was performed and why it is beneficial to incorporate it in the created system. The next section will describe the key difference between this chapter and \textit{chapter \ref{chapter:nonlinear}: \nameref{chapter:nonlinear}}, which is the way in which features are represented. Then, a dataset used for training and testing of the solution will be presented. The next section will provide information on how the results of semantic segmentation will be evaluated. Last, but definitely not least, an overview of the conducted experiments and their results will be provided.








