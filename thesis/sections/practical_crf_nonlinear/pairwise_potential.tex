The next component that is needed to compute the energy of the whole system is the pairwise potential and its definition needed to be changed as well in order to perform shape-based semantic segmentation. While the unary potential was responsible for detecting shapes, the role of the pairwise potential is to remove noises from the images. For this part of the system, feature function based on the Potts model was used. The general definition of this method was already presented in \textit{section \ref{sec:pariwise_potential}: \nameref{sec:pariwise_potential}}. Computation of the pairwise potential is based on differences between features of neighbouring superpixels from the original image, and not from the image after the process of colour quantisation as it was in case of the unary potential. There are four possible configurations of relations between a pair of superpixels. Two superpixels can be similar in terms of features and have the same label, which should be promoted, or a different label, which should be penalised. Similarly, two not similar superpixels can be assigned to the same class, which is a segmentation mistake, or to different classes, which is expected behaviour. Hence, such a feature function had to be proposed to model those relations between superpixel similarity and their labels at the same time. A chosen feature function returns a vector of two elements, which is different depending on whether the labels of a superpixel pair are the same or different, as in equation \ref{eq:nonlinear_potts_model_extended}.
\begin{align}
    \label{eq:nonlinear_potts_model_extended}
    \varphi(x_i,y_i,x_j,y_j) = \left\{\begin{matrix}
     \begin{bmatrix}
               \exp{\big(-\beta \left \| \phi(x_i) - \phi(x_j)\right \|^2\big)} \\
                1
            \end{bmatrix}  & , & y_i = y_j\\ 
    \\
    \begin{bmatrix}
               1 -\exp{\big(-\beta \left \| \phi(x_i) - \phi(x_j)\right \|^2\big)} \\
                1
            \end{bmatrix} & , & y_i \neq y_j
    \end{matrix}\right.
\end{align}

The second element of both vectors is a unit term needed for bias, and the first one defines similarity between neighbouring superpixels, which is dependent on the difference between their features $ \left \| \phi(x_i) - \phi(x_j)\right \|$. For experiments in this part of the system only one feature was chosen to compute the pairwise potential, and this feature is modelled as a three-dimensional vector of colours in CIELAB colour space. The difference between two colours is computed in terms of the Euclidean distance as in equation \ref{eq:colour_distance}. 
\begin{equation}
    \label{eq:colour_distance}
    \left \| \phi(x_i) - \phi(x_j)\right \|=\sqrt {(L_i-L_j)^2+(a_i-a_j)^2+(b_i-b_j)^2}
\end{equation}
To compute the pairwise potential also an image dependent parameter $\beta$ had to be established. It is calculated basing on on the average value of colour differences between superpixels for a given image.

Hence, knowing the definitions of the feature function for unary and pairwise potentials it was possible to perform the process of parameter learning. As the unary potential is defined just by one number, representing the probability of occurrence of a given label conditioned on superpixels features, and the pairwise potential is in a form of a two-dimensional vector, there are three weights to be learned by Stochastic Gradient Descent method. Those weights model the importance of both types of potentials for a given problem. Then, with a trained system, it was possible to perform the experiments by means of inference.
\newline The values of all hyperparameters that were required to perform segmentation in experiments presented in this chapter are shown in Table \ref{table:hyperparameters_nonlinear}
\begin{table}[ht]
    \caption{Values of hyperparameters required for the experiment on noise-free images.}
    \centering
    \begin{tabular}{|l|c|}
        \hline
        \rowcolor[HTML]{cecaca} 
        \textbf{Parameter name} & \textbf{Parameter value} \\ \hline
        Number of images (inputs) & 800 \\ \hline
        Number of states (outputs) & 4 \\ \hline
        Number of superpixels & 500 \\ \hline
        Training step &  0.0000001 \\ \hline
        Regularisation factor & 5000 \\ \hline
        Number of training epochs & 10 \\ \hline
        Convergence tolerance & 0.000005 \\ \hline
        Number of histogram bins & 17 \\ \hline
        Grid & $7 \times 7$ \\ \hline
        Neighbourhood size & 3 \\ \hline
    \end{tabular}
    \label{table:hyperparameters_nonlinear}
\end{table}