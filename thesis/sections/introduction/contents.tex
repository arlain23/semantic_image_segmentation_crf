This dissertation will be divided into two main parts, firstly all the theoretical concepts needed to understand the process of semantic image segmentation with the use of Conditional Random Fields will be provided, and next there will be a description of the system that was developed for this thesis, and the performed experiments.

An explanation of the main goal of the thesis will be presented in 
\textit{Chapter \ref{chapter:segmentation}: \nameref{chapter:segmentation}}. This chapter will include a description of what the semantic image segmentation is, with which methods it can be achieved and what are its possible applications. Next, \textit{chapter \ref{chapter:structured_prediction}: \nameref{chapter:structured_prediction}} will provide the theory behind the core of the created system in terms of procedures and algorithms required to fulfil its purpose. This chapter will begin with an explanation of how the input data are modelled for further processing. Then, the second part of the topic of this thesis will be described, which are Conditional Random Fields. Next section will be devoted to providing information on how the final prediction on an unknown sample is obtained. After that, there will be an explanation of the machine learning part of the system which is a parameter training of the model, which is based on images from the known dataset. The last section will present a way in which images are transformed into features representation that is understandable by the created model. 

\textit{Chapter \ref{chapter:system}: \nameref{chapter:system}} will provide an introduction to what are the goals of the developed system. Moreover, the implementation details in terms of the technological stack will be included and the key components of the system will be briefly described. \textit{Chapters \ref{chapter:linear}: \nameref{chapter:linear}} and \textit{\ref{chapter:nonlinear}: \nameref{chapter:nonlinear}} will be devoted to the experimental part of this dissertation. They will include a description of what the system is supposed to do and what were the steps needed for implementation. Furthermore, those chapters will contain information on which algorithms were chosen for the specific tasks, how the image dataset is transformed into meaningful data and finally what are the results of the semantic segmentation process. Last chapter, \textit{\nameref{chapter:conclusions}} will be devoted to the summary and comparison of the presented results, as well as, a description of possible improvements and extensions to the created system. 